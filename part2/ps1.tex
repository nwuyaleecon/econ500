%Jennifer Pan, August 2011

\documentclass[10pt,letter]{article}
	% basic article document class
	% use percent signs to make comments to yourself -- they will not show up.

\usepackage{amsmath}
\usepackage{amssymb}
\usepackage{enumitem}
	% packages that allow mathematical formatting
\DeclareMathOperator*{\argmax}{arg\,max}
\DeclareMathOperator*{\argmin}{arg\,min}
\usepackage{graphicx}
	% package that allows you to include graphics

\usepackage{setspace}
	% package that allows you to change spacing

\onehalfspacing
	% text become 1.5 spaced

\usepackage{fullpage}
	% package that specifies normal margins


\begin{document}
	% line of code telling latex that your document is beginning


\title{ECON500: Problem Set 1}

\author{Nicholas Wu}

\date{Fall 2020}
	% Note: when you omit this command, the current dateis automatically included

\maketitle
	% tells latex to follow your header (e.g., title, author) commands.


\section*{Problem 2}
\paragraph{(15.B.4)}
\begin{enumerate}[label=(\alph*)]
\item
\item
\item
\item
\item
\item
\end{enumerate}
\paragraph{(15.B.8)} Let the two utility functions be
\[ u_1(x,y) = x + f(y) \]
\[ u_2(x,y) = x + g(y) \]
By the second welfare theorem, any Pareto optimal allocation is supported by some prices and endowment choices. Normalize the numeraire price to 1. Let some interior Pareto optimal allocation be supported by the price vector $(1, p)$. By the optimization FOCs, since the allocation is interior, we require
\[ 1 - \lambda = 0 \]
\[ f'(y_1) - \lambda p =0 \]
or
\[ f'(y_1) = p \]
Similarly,
\[ g'(y_2) = p \]
Let the aggregate endowment of $y$ be $\omega_y$. Then at any Pareto optimal allocation, we must have
\[ f'(y_1) = g'(\omega_y - y_1) = p \]
\[ f'(y_1) - g'(\omega_y - y_1) = 0 \]
Define
\[ h(y) = f'(y) - g'(\omega_y - y) \]
Then
\[ h'(y) = f''(y) + g''(\omega_y - y) \]
Due to continuity and strictly convex preferences, both $f''$ and $g''$ are nonzero and have the same sign, and hence $h$ is a strictly monotonic function. Therefore, $h(y) = 0$ for at most one unique value of $y$, and since $y_1$ satisfies $h(y_1) = 0$, we know that since any interior Pareto optimal allocation must satisfy $h(y) = 0$, any interior Pareto optimal allocation always assigns $y_1$ to consumer 1 and $\omega_y - y_1$ to consumer 2.

\section*{Problem 3}
.... TODO smoothness ....

To show $x_i$ is a diffeomorphism, it suffices to argue the preimage map $x_i^{-1}$ is a well-defined function and smooth.
\section*{Problem 4}
The offer curve is $\mathcal{O}$

\end{document}
	% line of code telling latex that your document is ending. If you leave this out, you'll get an error
