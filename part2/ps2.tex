%Jennifer Pan, August 2011

\documentclass[10pt,letter]{article}
	% basic article document class
	% use percent signs to make comments to yourself -- they will not show up.

\usepackage{amsmath}
\usepackage{amssymb}
\usepackage{enumitem}
\usepackage{tikz}
	% packages that allow mathematical formatting
\DeclareMathOperator*{\argmax}{arg\,max}
\DeclareMathOperator*{\argmin}{arg\,min}
\usepackage{graphicx}
	% package that allows you to include graphics

\usepackage{setspace}
	% package that allows you to change spacing

\onehalfspacing
	% text become 1.5 spaced

\usepackage{fullpage}
	% package that specifies normal margins


\begin{document}
	% line of code telling latex that your document is beginning


\title{ECON500: Problem Set 1}

\author{Nicholas Wu}

\date{Fall 2020}
	% Note: when you omit this command, the current dateis automatically included

\maketitle
	% tells latex to follow your header (e.g., title, author) commands.


\section*{Problem 2}
\paragraph{(16.AA.1)}
Pick a sequence $(x^n, y^n) \to (x,y)$. By closure of $X, Y$, we have $x \in X$ and $y \in Y$. Then $\sum_i x_i^n = \omega + \sum_j y_j^n$, so by continuity $\sum_i x_i = \bar{\omega} + \sum_j y_j$. Hence $(x,y) \in A$, so $A$ is closed.
\paragraph{(16.AA.2)}
By assumption (i) $Y$ is closed, so $Y + \{ \bar{\omega} \} \cap \mathbb{R}^L_+ $ is closed. For compactness we just need to show boundedness. Suppose, for sake of contradiction, that $Y + \{ \bar{\omega} \} \cap \mathbb{R}^L_+ $ is unbounded. Pick a diverging sequence $y^n \in Y + \{ \bar{\omega} \} \cap \mathbb{R}^L_+ $. Then consider $||y^n - \bar{\omega}||$.
Since $y^n$ diverges, $||y^n - \bar{\omega}|| \to \infty$, and $y^n - \bar{\omega} \in Y$. Let $z^n = y^n - \bar{\omega}$. Then by convexity of $Y$, since $0 \in Y$, $z^n \in Y$, $z^n / ||z^n|| \in Y$. By closure of $Y$, the sequence $z^n / ||z^n|| \to z \in Y$. But $||z|| = 1$ (since the norm is continuous) and $z \in \mathbb{R}^L_+$, so we have a contradiction that $Y \cap \mathbb{R}^L_+ = \{ 0 \}$. So no such unbounded sequence can exist, and hence we have $Y + \{ \bar{\omega} \} \cap \mathbb{R}^L_+ $ is compact.

For the counterexamples, we show graphs of $Y$ for the 2-good case, where $Y + \{ \bar{\omega} \} \cap \mathbb{R}^L_+$ is not compact.

If $Y$ is not closed, the resulting set $Y + \{ \bar{\omega} \} \cap \mathbb{R}^L_+$ is not closed.

\begin{tikzpicture}
\draw[thick,->] (-4,0) -- (4,0) ;
\draw[thick,->] (0,-4) -- (0, 4) ;
\draw[dashed, red] (-3.5, 3.5) -- (3.5, -3.5);
\node at (0,0)[circle,fill,red,inner sep=1.5pt]{};
\end{tikzpicture}


If $Y$ is not convex:

\begin{tikzpicture}
\draw[thick,->] (-4,0) -- (4,0) ;
\draw[thick,->] (0,-4) -- (0, 4) ;
\draw[red] plot [smooth] coordinates {(-0.5, 3.5) (0,0) (3.5, -0.5)};
\end{tikzpicture}

If $0 \not \in Y$ :

\begin{tikzpicture}
\draw[thick,->] (-4,0) -- (4,0) ;
\draw[thick,->] (0,-4) -- (0, 4) ;
\draw[red] plot [smooth] coordinates {(-0.5, 4) (-0.5, -4)};
\end{tikzpicture}

If $Y \cap \mathbb{R}^L_+ \neq \{ 0 \}$  :

\begin{tikzpicture}
\draw[thick,->] (-4,0) -- (4,0) ;
\draw[thick,->] (0,-4) -- (0, 4) ;
\draw[red] plot [smooth] coordinates {(0.5, 4) (0.5, -4)};
\end{tikzpicture}
\paragraph{(16.AA.3)}
We first provide a lemma we will need later:

\textbf{Lemma}: Let $s_1^n$, $s_2^n$ be two divergent sequences, $||s_i^n|| \to \infty$, and suppose $s_i^n /||s_i^n|| \to s_i$, and $s_1 + s_2 \neq 0$. Then $||s_1^n + s_2^n|| \to \infty$.

\textbf{Proof}: Based on the given convergences, pick any value $K$. Since $s_i^n /||s_i^n|| \to s_i$, we have $|| s_1^n /||s_1^n|| + s_2^n /||s_2^n|| || \to ||s_1 + s_2||$, so there exists some $N$ such that for all $n > N$, $|| s_1^n /||s_1^n|| + s_2^n /||s_2^n|| || > \alpha || s_1 + s_2||$, $\alpha \in (0,1)$.
Since $s_1 + s_2 \neq 0$ and $s_i$ diverges, there exists some $N$ such that for all $n > N$, $||s_i^n|| > (1/\alpha) K / ||s_1 + s_2 ||$ for the same $\alpha \in (0,1)$.
Then $\forall n > N$, \[ || s^1_n + s^2_n|| \ge \min_i ||s^i_n|| || s_1^n /||s_1^n|| + s_2^n /||s_2^n|| || \ge \alpha || s_1 + s_2|| \left((1/\alpha) K / ||s_1 + s_2 || \right) = K\]
so $||s_1^n + s_2^n|| \to \infty$. $\blacksquare$.


Suppose, for sake of contradiction, that the given set is unbounded. Take a sequence $y_1^n$ such that $|| y_1^n || \to \infty$. Since $Y$ satisfies the properties of 16.AA.2, $y_1^n + y_2^n$ must be bounded. Consider $z_1^n = y_1^n/||y_1^n||$, $z_2^n = y_2^n/||y_2^n||$. Set $z_1^n \to z_1$, $z_2^n \to z_2$. Applying the contrapositive of the lemma implies that $z_1 + z_2 = 0$. Note by construction, $z_1, z_2$ have norm 1.
But $z_1$ is the limit point of the sequence $(y_1^n + y_2^1) /||y_1^n + y_2^1|| $ and by the fact that $0 \in Y$ and $Y$ is convex, we have that the sequence lies in $Y$, and hence by closure of $Y$, this implies $z_1 \in Y$. By the parallel argument, $z_2 \in Y$. But $z_1 + z_2 = 0$, $z_1, z_2 \in Y$, and $||z_1|| = 1 = ||z_2||$ a contradiction of the irreversibility assumption. Hence no such divergent sequence $y_1^n$ exists, and so we are done.
\section*{Problem 4}
\paragraph{(17.B.2)}
Suppose, for sake of contradiction, that the max of the excess demands converges. Fix some individual $i$ with nonzero limit endowment: $p \cdot \omega_i > 0$. Since the total excess demands converge, it must be the case that the excess demand of individual $i$ also converges. Let $z^*_i$ denote the limit excess demand of individual $i$. Denote the limit demand then as $x^*_i = z^*_i + \omega_i$. Consider any feasible bundle at price $p$, $x_i$. By feasibility, $p \cdot x_i \le p \cdot \omega_i$. Then
\[ p \cdot x_i + (p^n - p)\cdot x_i \le p \cdot \omega_i + (p^n-p) \cdot x_i \]
\[ p^n \cdot x_i \le p \cdot \omega_i + (p^n-p) \cdot x_i \]
\[ p^n \cdot x_i\left(\frac{p^n \cdot \omega_i}{p \cdot \omega_i + (p^n-p) \cdot x_i} \right) \le p^n \cdot \omega_i \]
So by the maximization of utility for demand, we have
\[ x_i(p^n) \succeq_i x_i \left(\frac{p^n \cdot \omega_i}{p \cdot \omega_i + (p^n-p) \cdot x_i} \right) \]
Since $p^n \to p$, the parenthesized expression on the RHS $\to$ 1, so the RHS $\to x_i$, and since $x_i = z_i(p^n) + w_i \to z^*_i + w_i = x^*_i$, we get by continuity of preferences
\[ x^*_i \succeq x_i \]
for all $x_i$ such that $p \cdot x_i \le p \cdot \omega_i$. But this is a contradiction of monotonicity of preferences: since $p_l = 0$, we can arbitrarily add good $l$ to bundle $x^*_i$ to get a bundle that is still affordable at $p$ but should yield higher utility from monotonicity. Hence, we have a contradiction, so the max of the excess demands cannot converge.

\section*{Problem 6}
We provide a counterexample. Take $K = [0,1]$, and let $\gamma(0) = (0,1)$ and $\gamma(k) = \{ 0.5 \} $ for any $k \in (0, 1]$. It is clear that $\gamma$ is a (constant) continuous function on $(0,1]$, so we just need to check upper hemicontinuity at 0. Take a sequence $k_n \in K$ converging to 0. Then $\gamma(k_n)$ only contains $0.5$ if $k_n \neq 0$, so we can always pick out a subsequence $k'_n$ such that $0.5$ is always in the image, and hence the image sequence converges to 0.5, and $0.5 \in \gamma(0)$. So $\gamma$ is upper hemicontinuous. It is immediate that $\gamma$ is not closed; $\gamma(0)$ is not closed.
\section*{Problem 7}
Fix an arbitrary $x$. We prove $\Gamma = \text{co} (\gamma)$ is upper hemicontinuous at $x$. Take an arbitrary neighborhood $V$ containing $\Gamma(x)$.  We need to show that $\exists$ a neighborhood $U$ of $x$, such that for any $x' \in U$, $\Gamma(x') \subseteq V$.

Suppose, for sake of contradiction, that in each neighborhood $U \ni x$, $\exists x^*_U$ such that $\gamma(x^*_U) \subseteq V$, but $\Gamma(x^*_U) \not \subseteq V$. This implies that we can select a sequence $x^*_1, x^*_2, ...$ which converges to $x$ such that $\gamma(x^*_i) \subseteq V$, but $\Gamma(x^*_i) \not \subseteq V$. For each $x^*_i$, pick $y_i \in \Gamma(x^*_i)$, $y_i \not \in V$. Since $y_i \in \Gamma(x^*_i)$, it is expressible as a convex combination of points in $\gamma(x^*_i)$.
Since $\gamma$ is compact-valued, $\gamma(x^*)$ is bounded, and hence $\Gamma(x^*_i) = \text{co} (\gamma(x^*))$ is also bounded. Then together, we have that $y_i$ is a bounded sequence, so by Bolzano-Weierstrass, we can select a convergent subsequence $z_j \to z$. Since $z_j \not\in V$, and $V$ is open, we have that $F \setminus V$ is closed, so $z \not \in V$, and since $\Gamma(x) \in V$, we have $z \not \in \Gamma(x)$.

Take the corresponding subsequence in $x^*_j$, which also must converge to $x$. By Caratheodory, since $z_j$ lies in the convex hull of $\gamma(x^*_j)$, we can define $d$ sequences of points and corresponding $d$ sequences of coefficients such that; $p^i_j \in \gamma(x^*_j)$ and $\sum_{i = 1}^d \lambda^i_j p^i_j = z_j$. Applying Bolzano-Weierstrass again, we can find some subsequences such that each $p^i_j$ converges. Since $\gamma$ is upper hemicontinuous, each subsequence of $p^i_j \in \gamma(x^*_j)$ converges to some $p^i \in \gamma(x)$. But this implies that $z_j$ converges to some $z$ in the convex hull of $\gamma(x)$, or $z \in \Gamma(x)$, contradicting what we just argued in the previous paragraph. Hence, it is impossible for every neighborhood $U$ of $x$ to have $\Gamma(x^*_U) \not \subseteq V$, and hence $\Gamma(x^*_U) \subseteq V$ for some $U \ni x$. Hence $\Gamma$ is upper hemicontinuous.

\section*{Problem 8}
We first prove a lemma.

\textbf{Lemma}: Any compact, convex set with nonempty interior $K \subseteq \mathbb{R}^d$ is isomorphic to the closed unit ball $B_n$ in some dimension $n$.

\textbf{Proof}: Select some point $c$ in the interior of $K$, and define $K'$ as $\{ k - c \ | \ k \in  K \}$, the translation of $K$ so that $c$ gets mapped to the origin. Consider the mapping $\varphi: K' \to B_n$
\[ \varphi(k) = \frac{k}{\max_{c \in \mathbb{R}^+, ck \in K'} ||ck||} \] Clearly, since $k \in K'$, the denominator is at least $||k||$, and hence $||\varphi(k)|| \le 1$, so $ \varphi(k) \in B_n$. Further, $\varphi$ is injective (two points along different rays cannot be mapped to the same point, and two points along the same ray have their ratio of magnitudes preserved) and surjective (any point in $B_n$ can be multiplied by $\max_{c \in \mathbb{R}^+, ck \in K'} ||ck||$ to recover a point in $K'$), so $\varphi$ is bijective. Finally, since $K'$ is convex, the denominator varies continuously, so $\varphi$ is continuous. Thus, $\varphi$ is an isomorphism between $K'$ and $B_n$, and since $K$ is isomorphic to $K'$ (translation is an isomorphism) we have that $K$ is isomorphic to $B_n$. $\blacksquare$ \\

Now, the generalization of Brouwer's follows from this lemma. Suppose $f: K \to K$ is continuous. If $K$ has empty interior, we can redefine a basis such that $K$ lies in a lower dimensional $\mathbb{R}^m$, $m < d$, and $K$ has nonempty interior in $\mathbb{R}^m$. Thus, it is without loss to consider $K$ with nonempty interior. By lemma 1, $K$ is isomorphic to the unit sphere $B_n$. Also by lemma 1, the $n+1$-simplex has nonempty interior after isomorphically projected into $\mathbb{R}^n$, so it is also isomorphic to $B_n$. These two facts together imply the $n+1$-simplex is isomorphic to $K$. Let $g: K \to S$ be an isomorphism from $K$ to the $n+1$-simplex $S$. Consider the composition $h = g\circ f \circ g^{-1}$. Then $h$ takes $S \to S$, and since composition of continuous functions is continuous, $h$ must be continuous. So by Brouwer's theorem on simplices, we have that $h$ has some fixed point $p \in S$. This implies
\[ (g \circ f \circ g^{-1}) (p) = p \]
\[ g^{-1}((g \circ f \circ g^{-1}) (p)) = g^{-1}(p)\]
\[ f(g^{-1}(p)) = g^{-1}(p) \]
Then since $g^{-1}(p) \in K$, we have that $g^{-1}(p)$ is a fixed point of $f$. Hence we have extended Brouwer to an arbitrary compact convex domain $K$.
\section*{Problem 9}
\textit{Collaborator: Jingyi Cui, but solutions independently written}

Take an arbitrary labelling of the vertices $v_0, v_1, ... v_n$ of the simplex $S$. Define $\phi$ as the shift-by-one mapping, or \[ \phi(i) = i + 1 \mod {n+1} \]
We construct a continuous mapping $\varphi$ as follows. Consider an arbitrary $s \in S$. Define the subdivision carrier $\chi^*(s)$ as the vertices of the smallest simplex in the subdivision $\{ S_i \}$ that contains $s$, which exists by definition of subdivision. Let $L_i$ denote the labelling map of the vertices of $S_i$, that is it maps each vertex $v$ of $S_i$ to a number $0, ... n$. Let the vector of barycentric coordinates $\vec{\lambda}^{S_i}_s$ of $s$ with respect to the vertices of the subdivision $S_i$; specifically,
\[ s = \sum_{v \in \chi^*(s) } \lambda^{S_i}_s(v) v  \]
Then define
\[ \varphi(s) = \sum_{w \in \chi^*(s) } \lambda^{S_i}(w) v_{\phi(L(w))} \]
Intuitively, we are mapping each subdivision onto the subsimplex determined by the labelling of its vertices (preserving barycentric coordinates), and permuting all the vertices in a derangement. Note that for the vertices $v_j$ of $S$, $\varphi(v_j)$ maps $v_j$ to $v_{j+1}$ if $j < n$ and to $v_0$ if $j=n$. Now we need a couple of important properties of $\varphi$, that we will prove. We want to be able to argue that the operation of $\varphi$ cannot fix the boundary, and since $\varphi$ can only map points from a fully labeled subdivision into the interior of $S$, the fixed point that exists by Brouwer must be interior, and therefore since any interior point in the image must lie inside a fully labeled subdivision, a fully labeled subdivision exists. We do this formally by proving two lemmas:

\textbf{Lemma 1}: $\varphi$ has no fixed points on the boundary of the simplex $S$.

\textbf{Proof}:  Pick any arbitrary point $p$ on the boundary of $S$. Since $p$ is on the boundary, the carrier $\chi(p)$ does not contain all the vertices of $S$. Further, by definition of carrier, $p$ can be written as a combination of the vertices of $\chi(p)$,
\[ p = \sum_{w \in \chi(p)} \mu_p(w)w  \]
where $\mu_p(w) > 0$ for each $w \in \chi(p)$. Then, we have by definition of $\varphi$,
\[ \varphi(p) = \sum_{w \in \chi^*(p) } \lambda^{S_i}(w) v_{\phi(L(w))} \]
Since $L$ is admissible, $v_{L(w)} \in \chi(p)$. Since $p$ is a boundary point, $\chi(p)$ does not contain all the vertices of $S$, and hence the set $\{ v_{\phi(L(w))} \ | \ w \in \chi^*(p)\} \neq \chi(p) $. Therefore, expressing in terms of barycentric coordinates, there exists some barycentric coordinate direction $v^*$ such that only one of $p$ or $\varphi(p)$ has a nonzero barycentric coordinate in the direction of $v^*$. Hence, we cannot have $p = \varphi(p)$, so $p$ cannot be a fixed point of $\varphi$. $\blacksquare$ \\

\textbf{Lemma 2}: $\varphi$ is continuous.

\textbf{Proof}: It suffices to argue that $\varphi$ is continuous within the closure of each subdivision $cl(S_i)$. Then, since the union of the closures of $S_i$ covers the simplex $S$ by the definition of a valid subdivision, this implies that $\varphi$ is continuous on $S$.

Consider a sequence of points $s_n \to s \in cl(S_i)$, where each $s_n \in cl(S_i)$. Consider the barycentric coordinates of $s_n$ with respect to the vertices of $S_i$ as $\mu_n^{S_i}$, and let $s$ have barycentric coordinates $\mu^{S_i}$ with respect to $S_i$. Since coordinate mappings are continuous, and $s_n \to s$, we must have $\mu_n^{S_i} \to \mu^{S_i}$. But since the labeling $L$ is fixed for the vertices of $S_i$, we thus have \[ \sum_{w \in S_i} \mu_n^{S_i}(w) \cdot v_{(\phi\circ L)(w)} \to \sum_{w \in S_i} \mu^{S_i}(w) \cdot v_{(\phi\circ L)(w)} \]  \[ \varphi(s_n) \to \varphi(s) \ \blacksquare\] \\

Now, by Brouwer, we have that since $\varphi$ is a continuous mapping $S \to S$ by Lemma 2, we know that there exists a fixed point $p$ of $\varphi$. Now, by Lemma 1, $p$ must be in the interior of $S$. Let $\lambda^S_p$ be the barycentric coordinates of $p$ with respect to the vertices of $S$. We then note that since $p$ is interior, $\lambda^S_p > 0$. Further, since $p$ is a fixed point of $\varphi$, we have
\[  \sum_{w \in \chi^*(s) } \lambda^{S_i}(w) v_{\phi(L(w))} = \varphi(p) = p = \sum_{j = 0}^n \lambda^S(v_j) v_j \]
Then since each of the $v_i$ are linearly independent and form a basis for the simplex $S$, we can match sum terms in the projection in the direction of each $v_j$. Specifically,
\[ \lambda^S(v_j) = \lambda^{S_i}(w_j)  \]
where $w_j$ is such that $\phi(L(w_j)) = j$, or $L(w_j) = j-1 \mod (n+1)$. Since $p$ exists, each $w_j$ must exist, and therefore $(L(w_1), L(w_2), ... L(w_n), L(w_0)) = (0, 1, ... n)$ in that order. Hence, this means that the subdivision $S_i$ containing $p$ is fully labeled, and thus Sperner's lemma holds.
\end{document}
	% line of code telling latex that your document is ending. If you leave this out, you'll get an error
