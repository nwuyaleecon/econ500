%Jennifer Pan, August 2011

\documentclass[10pt,letter]{article}
	% basic article document class
	% use percent signs to make comments to yourself -- they will not show up.

\usepackage{amsmath}
\usepackage{amssymb}
	% packages that allow mathematical formatting
\DeclareMathOperator*{\argmax}{arg\,max}
\DeclareMathOperator*{\argmin}{arg\,min}
\usepackage{graphicx}
	% package that allows you to include graphics

\usepackage{setspace}
	% package that allows you to change spacing

\onehalfspacing
	% text become 1.5 spaced

\usepackage{fullpage}
	% package that specifies normal margins


\begin{document}
	% line of code telling latex that your document is beginning


\title{ECON500: Problem Set 6}

\author{Nicholas Wu}

\date{Fall 2020}
	% Note: when you omit this command, the current dateis automatically included

\maketitle
	% tells latex to follow your header (e.g., title, author) commands.


\section*{Part 1}
\paragraph{(1.1)} Free disposal implies that if $y'_i \le y_i$ $\forall i$ and $y \in Y$, then $y' \in Y$.

Suppose $Y$ is closed, convex, and contains the negative orthant. Suppose $y'_i \le y_i$ and $y \in Y$. Define the vector $\Delta$ as $\Delta_i = y_i' - y_i$. Then by our supposition, $\Delta_i \le 0$, so $\Delta$ lies in the negative orthant, and therefore there exists $\lambda > 1$ such that $y + \lambda \Delta$ lies in the negative orthant. This implies $y + \lambda \Delta \in Y$. Then by convexity:
\[  y' = \Delta + y = \frac{1}{\lambda}(y + \lambda \Delta) + \frac{\lambda - 1}{\lambda} y \in Y  \]
Hence $y' \in Y$, so $Y$ satisfies free disposal.
\paragraph{(1.2)}
Decreasing returns to scale:
\[ y \in Y \implies \alpha y \in Y \ \alpha \in [0,1] \]
Increasing returns to scale:
\[ y \in Y \implies \alpha y \in Y \ \alpha \in [1,\infty) \]
Constant returns to scale:
\[ y \in Y \implies \alpha y \in Y \ \alpha \in [0,\infty) \]

Suppose $Y$ exhibits constant returns to scale. Then for $y \in Y$, $\alpha \in [0,1]$, by definition of constant returns $\alpha y \in Y$, so $Y$ exhibits decreasing returns to scale. Similarly, if $\alpha \in [1, \infty)$, $\alpha y \in Y$, so $Y$ also exhibits increasing returns to scale. In the reverse, if $Y$ exhibits increasing and decreasing returns to scale, then for $y \in Y$, for any $\alpha \in [0,1] \cup [1, \infty) = [0, \infty)$, $\alpha y \in Y$, so $Y$ exhibits constant returns to scale.

Decreasing returns is not equivalent to convexity of $Y$. Consider $Y = \{ (x,y) \ : \ xy < 1 \text{ or } x < 0 \text{ or }  y < 0\}$. This exhibits decreasing returns: if $x,y \in Y$, $\alpha \in [0,1]$, then $(\alpha x)(\alpha y) = \alpha^2xy < \alpha^2 \le 1 $, so $\alpha (x,y) \in Y$. However, this is not convex: $(0,3), (3, 0) \in Y$, but $(1.5, 1.5) \not \in Y$.

Convexity of $Y$ does imply decreasing returns, however. If $Y$ is convex, then for any $y \in Y$, $\alpha \in [0,1]$, $\alpha y + (1-\alpha) 0 = \alpha y \in Y$.

Increasing returns is not equivalent to the convexity of $Y^c$. Consider $Y = \{ (x,y) \ : \ xy < 1 \text{ or } x < 0 \text{ or }  y < 0\}$. $Y^c = \{ (x,y) \ : \ x \ge 0, y \ge 0, xy \ge 1\} $ is convex. However, $Y$ does not exhibit increasing returns: $(0.5, 0.5) \in Y$, but $(2,2) \not \in Y$.

Increasing returns does not even imply the convexity of $Y^c$. Consider $Y = \{ (x,y) \ : x < 0, y < 0 \}$ (the negative orthant). Then clearly $Y$ satisfies increasing returns: if $y \in Y$, then $\alpha y \in Y$. However, $Y^c$ is not convex: $(-3, 1)$ and $(1, -3)$ are not in $Y$, but their midpoint $(-1, -1)$ is in $Y$.

Now, suppose there are $L$ goods and the production set $Y$ exhibits constant returns to scale. Fix one of the inputs at $y^* < 0$, and consider the production set $Y'$ over the remaining $L-1$ goods. Suppose $(y_1, y_2, ... y_{L-1}) \in Y'$, and $\alpha \in [0,1]$. This implies that
\[ (y_1, y_2, ... y_{L-1}, y^*) \in Y \]
By constant returns to scale on $Y$,
\[ (\alpha y_1, \alpha y_2, ... \alpha y_{L-1}, \alpha y^*) \in Y \]
By free disposal, $\alpha y^* > y^*$, so
\[ (\alpha y_1, \alpha y_2, ... \alpha y_{L-1}, y^*) \in Y \]
Then
\[ (\alpha y_1, \alpha y_2, ... \alpha y_{L-1}) \in Y' \]
and hence $Y'$ exhibits decreasing returns.
\paragraph{(1.3)}
Increasing returns: for $\alpha \ge 1$,
\[ F(y) \le 0 \implies F(\alpha y) \le 0 \]
Decreasing returns: for $\alpha \in [0,1]$,
\[ F(y) \le 0 \implies F(\alpha y) \le 0 \]
\paragraph{(1.4)}
We first show that increasing returns implies failure of the SOCs for profit maximization. Suppose that $Y$ satisfies increasing returns. Let $F(y)=0$ denote the frontier of $Y$. Then the maximization problem is
\[ \max p \cdot y \]
\[ F(y) = 0 \]
The Lagrangian is
\[ \mathcal{L} = p \cdot y - \lambda F(y) \]
Consider an arbitrary critical point, $y^*$. By the FOCs,
\[ p = \lambda \nabla F(y^*) \]
\[ F(y^*) = 0 \]
Consider $y' = (1+\epsilon)y^*$ for some small $\epsilon > 0$. By increasing returns, $F(y') \le 0$. However, $p \cdot y' - p \cdot y^* = \epsilon ( p \cdot y^*) > 0$ since $p \cdot y^* \ge 0$ as $0 \in Y$. This implies that $y^*$ cannot maximize the objective, and hence the second order conditions must fail.

Now, consider the production set $Y = \{(x,y) \ : \ x + y \le 0 \}$. Clearly, this set exhibits decreasing returns. Suppose prices are $(1, 2)$. There is no solution to the profit maximization: for $k > 0$, $(-k, k) \in Y$, and $p \cdot (-k, k) = k$ is arbitrarily increasing in $k$.
\paragraph{(1.5)}
Suppose $Y = \{ (-z, q) \ : f(z) \ge q \} $. Let the prices be given by $w$, and the cost function is the minimum value of
\[ \min w \cdot z \]
subject to
\[ q \le f(z) \]
Let $z(w,q)$ be the minimizer, $c(w,q)$ be the minimum value.
Increasing costs implies that for $\alpha \ge 1$,
\[ \alpha c(w, q) \le  c(w,\alpha q) \]
Decreasing costs implies that for $\alpha \ge 1$
\[ \alpha c(w, q) \ge  c(w,\alpha q) \]
And constant costs implies that for $\alpha \ge 1$
\[ \alpha c(w, q) =  c(w,\alpha q) \]

We now relate this cost function to the features of the set $Y$ in terms of increasing/decreasing/constant returns.
Suppose $Y$ satisfies increasing returns. Pick a $w$, and consider $(-z(w,q), q) \in Y$. Let $\alpha \ge 1$. By increasing returns $(-\alpha z(w,q), \alpha q) \in Y$. Then we know $f(\alpha z(w,q)) \ge \alpha q$. So $c(w, \alpha q) \le w \alpha z(w,q) = \alpha c(w,q)$. Hence increasing returns implies decreasing costs.

Similarly, we show that decreasing returns implies increasing costs. Suppose $Y$ satisfies decreasing returns. Take $\alpha \ge 1$. Pick a $w$, and consider $(-z(w,\alpha q), \alpha q) \in Y$. By decreasing returns $(-\frac{1}{\alpha}z(w,\alpha q), q) \in Y$. Therefore
\[ \frac{1}{\alpha} w \cdot z(w, \alpha q) \ge c(w, q) \]
\[ \frac{1}{\alpha} c(w, \alpha q) \ge c(w, q) \]
\[ c(w, \alpha q) \ge \alpha c(w, q) \]
so we have increasing costs.

Clearly, we can observe that if we have constant returns, we have both increasing and decreasing returns, and hence increasing and decreasing costs. This implies we have constant costs.

For the opposite direction, if we suppose that the upper contour sets of $f$ are convex, we can show the converses of the above statements. Suppose we have decreasing costs, and the upper contour sets of $f$ are convex. Consider $(-z, q) \in Y$. This implies $f(z) \ge q$, and hence $\forall w$, $w \cdot z \ge c(w, q)$. Now, using the upper contour set convexity, we showed on the previous pset that
\[ Y = \{ (-z, q) \ : \ wz \ge c(w,q) \}  \]
Then by using decreasing costs, we get that fixing $\alpha > 1$, $\forall w$,
\[ \alpha w \cdot z \ge \alpha c(w,q)\ge c(w, \alpha q) \]
\[ w \cdot (\alpha z) \ge c(w, \alpha q) \]
Hence, using the alternate definition of $Y$, we get that $(-\alpha z, \alpha q) \in Y$. Hence $Y$ satisfies increasing returns.

Now, suppose we have increasing costs. Once again, consider $(-z, q) \in Y$. Consider $z' = z/\alpha,  q' = q/\alpha)$, $\alpha > 1$. It suffices to show for decreasing returns that $(-z', q') \in Y$. Suppose $(-z', q') \not \in Y$. Using the alternate definition of $Y$ we showed on the previous pset, this implies there exists some $w$ such that
\[ w z' < c(w, q') \]
Using increasing costs,
\[ \alpha w z' < \alpha c(w, q') \le c(w, \alpha q') = c(w, q) \]
\[ w z < c(w,q) \]
But this implies that $(-z, q) \not \in Y$, a contradiction. Hence we must have $(-z', q' ) \in Y$, and therefore we have decreasing returns.

From the previous parts, we showed (assuming the upper contour sets of $f$ are convex) that increasing costs implies decreasing returns and decreasing costs implies increasing returns. Trivially then, constant costs implies both increasing and decreasing costs, which imply both increasing and decreasing returns, which together imply constant returns. Hence constant costs implies constant returns.
\paragraph{(1.6)}
Let the firm have production function $f$, and face prices $p_1, p_2, q$. The maximization problem is
\[ \max qy - p_1 x_1 - p_2 x_2 \]
\[ y \le f(x_1, x_2) \]
From FOCs:
\[ q - \lambda = 0 \]
\[ \lambda f_1(x_1, x_2) = p_1 \]
\[ \lambda f_2(x_1, x_2) = p_2 \]
Differentiating in $p_1$, we have
\[ qf_{11}(x_1, x_2) \frac{\partial x_1}{\partial p_1} + qf_{12}(x_1, x_2) \frac{\partial x_2}{\partial p_1} = 1 \]
\[ qf_{21}(x_1, x_2) \frac{\partial x_1}{\partial p_1} + qf_{22}(x_1, x_2) \frac{\partial x_2}{\partial p_1} = 0 \]
Together, in matrix form, we have
\[ \begin{bmatrix}
qf_{11}(x_1, x_2) & qf_{12}(x_1, x_2) \\
qf_{12}(x_1, x_2) & qf_{22}(x_1, x_2)
\end{bmatrix} \begin{bmatrix} \frac{\partial x_1}{\partial p_1} \\ \frac{\partial x_2}{\partial p_1}\end{bmatrix} = \begin{bmatrix} 1 \\ 0 \end{bmatrix} \]
Applying Cramer's rule,
\[ \frac{\partial x_1}{\partial p_1} = \frac{1}{|H|} (qf_{22}(x_1, x_2) ) \]
\[ \frac{\partial x_2}{\partial p_1} =  -\frac{1}{|H|} (qf_{12}(x_1, x_2) ) \]
Then since at maximization, $y = f(x_1, x_2)$,
\[ \frac{\partial y}{\partial p_1} = f_1(x_1, x_2)\frac{\partial x_1}{\partial p_1} + f_2(x_1, x_2)\frac{\partial x_2}{\partial p_1} = \frac{q}{|H|} \left(f_1(x_1, x_2)f_{22}(x_1, x_2) - f_2(x_1, x_2)f_{12}(x_1, x_2) \right)  \]

In order for this sign to be negative, we need
\[ f_1(x_1, x_2)f_{22}(x_1, x_2) < f_2(x_1, x_2)f_{12}(x_1, x_2)) \]
For this sign to be positive, we need
\[ f_1(x_1, x_2)f_{22}(x_1, x_2) > f_2(x_1, x_2)f_{12}(x_1, x_2)) \]
This is distinct from Giffen; in Giffen, we are interested in an own-effect, where here we require conditions on the cross derivative of $f$ on goods 1 and 2.
\paragraph{(1.7)}
$x_1$ and $x_2$ are complements if
\[ \frac{\partial x_1}{\partial p_2} < 0 \]
\[ \frac{\partial x_2}{\partial p_1} < 0 \]
By monotone comparative statics, we will have the desired complementary behavior if $qf(x_1, x_2) - p_1 x_1 - p_2 x_2$ exhibits increasing differences in $x_1, -p_2$ and increasing differences in the pair $x_1, x_2$. We quickly observe that the first condition trivially holds, and increasing differences in the second pair is the critical observation.

Using the implicit function theorem and the results from the previous part, we get that we have complementary behavior if
\[ f_{12}(x_1, x_2) \ge  0 \]
We note that this is sufficient for increasing differences in $x_1, x_2$, and aligns with our answer from monotone comparative statics.


\paragraph{(1.8)}
\begin{itemize}
\item Let the production technology be characterized by $f$, and let us split the price vector into $p, w$ for outputs and inputs, respectively. Then we can rewrite the constrainted optimization as the unconstrained:
\[ \max u(p \cdot f(z) - w\cdot z) \]
The FOCs are
\[ u'(p\cdot f(z) - w\cdot z)(p\cdot f_i(z) - w_iz_i) = 0 \]
Now assuming $u$ is a strictly increasing function, we get that this condition is equivalent to
\[p\cdot f_i(z) - w_iz_i = 0 \]
And hence this is the same FOC as profit maximization. Likewise, for the second derivatives, we note that the only effect is the entire bordered Hessian is multiplied by the nonzero, positive quantity $u'(p\cdot f(z) - w\cdot z)$, and so the same SOCs hold. Once again, due to convexity of the maximum function, we have that
\[ (u(p \cdot f(z) - w \cdot z) - u(p \cdot f(z') - w \cdot z')) \ge 0 \]
\[ (u(p' \cdot f(z') - w' \cdot z') - u(p' \cdot f(z) - w' \cdot z)) \ge 0 \]
If $u$ is increasing, this implies
\[ ((p \cdot f(z) - w \cdot z) - (p \cdot f(z') - w \cdot z')) \ge 0 \]
\[ ((p' \cdot f(z') - w' \cdot z') - (p' \cdot f(z) - w' \cdot z)) \ge 0 \]
Hence,
\[ ((p, -w) \cdot (f(z), z) - (p, -w) \cdot (f(z'), z') - (p', -w') \cdot (f(z), z) + (p', -w') \cdot (f(z'), z') \ge 0 \]
\[ ((p, -w) - (p', -w')) \cdot (f(z), z) - (f(z'), z')) \ge 0 \]
We can of course take $p', -w' \to (p,w)$ to establish the differential result. Thus, the law of supply holds.
\item We require $u$ to be increasing and nonsatiated in order for the supply functions to be unchanged by including utility in the model. If $u$ satisfies these conditions, then
\[ u(a) \ge u(b) \iff a \ge b \]
and hence $y(p) = \arg \max u(p y) = \arg \max py $. We can also observe this through the FOC:
\[ u'(p\cdot f(z) - w\cdot z)(p\cdot f_i(z) - w_iz_i) = 0 \]
If $u'(p\cdot f(z) - w\cdot z) > 0$, then we have the same supply and conditional supply functions as the plain profit maximization ($u(\pi) = \pi$).

Now, we show that under these assumptions on $u$ we have both Shepard's lemma and Hotelling's lemma. Specifically, by the envelope theorem, we get that
\[ u'y = \frac{\partial u(\pi)}{\partial \pi} \frac{\partial \pi}{\partial p}  \]
\[ u'y =  u' \frac{\partial \pi}{\partial p}\]
\[ y = \frac{\partial \pi}{\partial p} \]
So we have Hotelling's. Considering an input, we find
\[ u'z = \frac{\partial u(\pi)}{\partial e} \frac{\partial e}{\partial w}  \]
\[ u'z = \frac{\partial u(\pi)}{\partial e} \frac{\partial e}{\partial w}  \]
\[ u'z = u' \frac{\partial e}{\partial w}  \]
\[ z = \frac{\partial e}{\partial w}  \]
and hence we have Shepard's.


\item We can explain $u$ as the owner's utility of wealth. We have shown already that as long as $u$ is strictly increasing, the profit maximization condition is equivalent to the utility maximization for a firm owner.
\end{itemize}
\pagebreak
\section*{Part 2}
\paragraph{(2.1)}
Consider the lotteries $\delta_x$, $x \in X$, which give outcome $x$ with probability $1$. Since there are finite $\delta_x$ and preferences are transitive and complete, there exists some $y,z$, such that among these $\delta_x$, $\delta_y$ is best and $\delta_z$ is worst. Now, we argue that $\delta_y$ is best among all lotteries, and $\delta_z$ is worst among all lotteries.

Consider an arbitrary lottery $p$, which assigns probabilities $p(x)$ to outcome $x$, and $\sum_{x \in X} p(x) = 1$. Then since for all $x \in X$,
\[ \delta_y \succeq \delta_x \]
Let us enumerate the $x$s, and WLOG we set $y = x_n$, $z = x_1$. By repeatedly applying independence,
\[ \delta_{x_n} = p(x_1) \delta_{x_n} + (1-p(x_1))\delta_{x_n}  \succeq  p(x_1) \delta_{x_1} + (1-p(x_1))\delta_{x_n}  \]
\[ \succeq p(x_1) \delta_{x_1} + (1-p(x_1))\left( \frac{p(x_2)}{1-p(x_1)} \delta_{x_2} + \frac{1 - p(x_1) - p(x_2)}{1-p(x_1)}\delta_{x_n} \right)  \]
\[ = p(x_1) \delta_{x_1} + p(x_2) \delta_{x_2} +  (1 - p(x_1) - p(x_2))\delta_{x_n} \]
\[ ... \succeq \sum_{i=1}^k p(x_i)\delta_{x_i} + \left(1 - \sum_{i=1}^k p(x_i)\right) \delta_{x_n} \]
\[ ... \succeq \sum_{i=1}^k p(x_i)\delta_{x_i} + \left(1 - \sum_{i=1}^k p(x_i)\right) \left( \frac{p(x_{k+1})}{1 - \sum_{i=1}^k p(x_i)} \delta_{x_k} + \frac{1 - \sum_{i=1}^{k+1} p(x_i)}{1 - \sum_{i=1}^k p(x_i)} \delta_{x_n} \right) \]
and hence from induction we find
\[ \delta_{x_n} \succeq \sum_{i=1}^n p(x_i)\delta_{x_i} = p \]
Thus, $\delta_{x_n}$ is the best lottery. Similarly, since $\delta_x \succeq \delta_z$ for all $x \in X$, enumerating such that $x_1 = z$, we can repeatedly apply independence:
\[ \delta_{x_1} = p(x_n) \delta_{x_1} + (1-p(x_n))\delta_{x_1}  \preceq  p(x_n) \delta_{x_n} + (1-p(x_1))\delta_{x_1}  \]
\[ \preceq p(x_n) \delta_{x_n} + (1-p(x_n))\left( \frac{p(x_{n-1})}{1-p(x_n)} \delta_{x_{n-1}} + \frac{1 - p(x_n) - p(x_{n-1})}{1-p(x_n)}\delta_{x_1} \right)  \]
\[ = p(x_n) \delta_{x_n} + p(x_{n-1}) \delta_{x_{n-1}} +  (1 - p(x_n) - p(x_{n-1}))\delta_{x_1} \]
\[ ... \preceq \sum_{i=0}^k p(x_{n-i})\delta_{x_{n-i}} + \left(1 - \sum_{i=0}^k p(x_{n-i})\right) \delta_{x_1} \]
\[ ... \preceq \sum_{i=0}^k p(x_{n-i})\delta_{x_{n-i}} + \left(1 - \sum_{i=0}^k p(x_{n-i})\right) \left( \frac{p(x_{n-k-1})}{1 - \sum_{i=0}^k p(x_{n-i})} \delta_{x_k} + \frac{1 - \sum_{i=0}^{k+1} p(x_{n-i})}{1 - \sum_{i=0}^k p(x_{n-i})} \delta_{x_1} \right) \]
and hence from induction we find
\[ \delta_{x_1} \preceq \sum_{i=1}^n p(x_i)\delta_{x_i} = p \]
So $\delta_{x_1}$ is the worst lottery.

\paragraph{(2.2)}
We first show independence implies betweenness. Suppose we have independence, and $p \sim q$. Then for $\alpha \in [0,1]$
\[ p = \alpha p + (1-\alpha) p \sim \alpha p + (1-\alpha) q  \]
where the second part follows from independence (taking $r = p$). Hence betweenness is implied by independence.

To argue that betweenness does not necessitate independence, consider the Allais paradox. Consider a preference over lotteries $(p, q, 1-p-q)$ on $(5,1,0)$. Suppose a consumer's utility over a lottery $(p,q)$ is given by
\[u(p,q) = \frac{1.01 + p}{1.01 - p - q}    \]
Note then, for any $\alpha$ if $u = u(p,q) = u(p',q')$, the lottery $ (\alpha p + (1-\alpha) p', \alpha q + (1-\alpha)q' ) $ satisfies
\[ \frac{1.01 + (\alpha p + (1-\alpha)p')}{1.01 - (\alpha p + (1-\alpha) p') - (\alpha q + (1-\alpha)q')} = \frac{1.01+ (\alpha p + (1-\alpha)p')}{\alpha(1.01 - p - q) + (1-\alpha)(1.01-p-1)} \] \[ = \frac{1.01+ (\alpha p + (1-\alpha)p')}{\alpha(1.01 + p)/u + (1-\alpha)(1.01 + p')/u} = u \frac{1.01+(\alpha p + (1-\alpha)p')}{1.01+ \alpha p + (1-\alpha) p'} = u  \]
and hence these preferences satisfy betweenness. We now note that this can accommodate Allais. Specifically, if we let $p$ be the probability of $5$, $q$ the probability of $1$, then our preferences say that
\[ u(0,1) = \frac{1.01}{.01} = 101 \]
\[ u(.1, .89) = \frac{1.11}{.02} = 55.5 \]
And hence $A \succ B$. Additionally,
\[ u(0, .11) =\frac{1.01}{0.9} = 1.122... \]
\[ u(.1, 0) = \frac{1.11}{.91} = 1.219...\]
So $D \succ C$. Hence, these preferences, which satisfy betweenness, accommodate Allais, while independence cannot accommodate Allais. Therefore, there exist situations where betweenness holds, but not independence.
\paragraph{(2.3)}
\begin{itemize}
\item The utility maximization problem is:
\[ \max_x \sum_\theta p(\theta) u(\theta, x) \]
\item The utility maximization given the state revelation of $\theta$ is
\[ \max_x u(\theta, x) \]
Suppose the respective argmax is given by $x^*(\theta)$, and the argmax of the previous problem is $x^*$. Then the expected utility assuming actions occur after state revelation is
\[ \sum_\theta p(\theta) u(\theta, x^*(\theta)) \ge \sum_\theta p(\theta) u(\theta, x^*) \]
where we have just used that $u(\theta, x^*(\theta)) \ge u(\theta, x^*)$, which follows from the properties of $x^*(\theta)$. But the RHS of this is the expected utility of acting before state revelation. Hence, we have the expected utility is greater if we can act after state revelation rather than acting before.

We construct an example where the inequality is strict. Let $\theta \in \{ 0, 1\}$ with equal probability, and suppose the actions are $x \in [0,1]$. Define $u(\theta, x) = -(x - \theta)^2$. If we can pick $x$ after observing $\theta$, we clearly just take $x = \theta$, and our expected utility is $0$. However, if we must commit to $x$ before observing $\theta$, our utility maximization is
\[ \max -\frac{1}{2}x^2 - \frac{1}{2}(x-1)^2 = \max - x^2 + x - 1/2 \]
This is maximized at $x = 1/2$, which gives us expected utility $-1/4 < 0$.
\item The action after state revelation allows us to always achieve better utility, sometimes strictly. Hence, we can interpret information about the state as increasing our expected utility, and we might claim in this situation that more information is better.
\end{itemize}
\paragraph{(2.4)}
\begin{itemize}
\item Your expected salary is
\[ \int_{\underline{z}}^{\overline{z}} z f(z) \ dz \]
and the expected utility is
\[ \int_{\underline{z}}^{\overline{z}} u(z) f(z) \ dz \]
\item By Jensen's inequality, since $u$ is strictly concave,
\[ u(E[z]) > E[u(z)]  \]
Hence, your utility from simply taking the expected salary is better than your expected utility if you take the test and reveal your score. Hence, you wouldn't want to take the exam.
\item If you know your score, then you clearly just take the exam if $z \ge E[z]$, because as long as $u$ is increasing, your subsequent utility satisfies $u(z) \ge u(E[z])$. Therefore, the people who don't take the exam all must have $z < E[z]$, so the average score among all people not taking the exam must be less than $E[z]$.

Now, suppose the salary given to people who don't take the exam is the average score of people who don't take the exam. We prove that the only people who won't take the exam have $z = \underline{z}$. Suppose the average score of people who don't take the exam is $\mu_z$ (i.e. someone with $z > \underline{z}$ doesn't take the exam). Everyone with score greater than $\mu_z$ will want to take the exam, but this implies that the average of people who don't take the exam must be at most $\mu_z$. This implies that everyone who takes the exam has score $z = \mu_z$, and the only way for that to be possible is if $\mu_z = \underline{z}$. Hence, the only people who don't take the exam have score $\underline{z}$ (to them, it does not matter whether they take the exam or not), and everyone with a score higher than that will take the exam. Anyone not taking the exam receives a salary of $\underline{z}$.
\item The outcome of this vote depends on the distribution $f$. Let $z'$ denote the median score. If $z' > E[z]$, then a majority of students will want to take their exam, because they would get a score higher than $E[z]$ and hence a higher salary. If $z' < E[z]$, then a majority of students will not want to take the exam, since half of the students would expect to receive a higher salary $E[z]$ than what they would get if they took the exam.
\item The students \textbf{on aggregate} do achieve a higher utility from receiving information about their score. If they all know that most of them would be better off from not taking the exam, then they will vote to not take the exam, and more than half the class would be better off. If the exam is taken, then this is the same result as the scenario previously where only the students with the lowest possible score do not take the exam, and no welfare is lost
\end{itemize}
\paragraph{(2.5)}
I'd not want to know. You can modify the model by taking the actor as being risk-loving in some sense, either by assigning a utility value to the uncertainty or taking a transformed expected utility rather than a vNM utility.
\end{document}
	% line of code telling latex that your document is ending. If you leave this out, you'll get an error
